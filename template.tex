\documentclass[11pt,a4paper, sans]{moderncv}

\moderncvstyle{casual}  % style options are 'casual' (default), 'classic', 'oldstyle' and 'banking'
\moderncvcolor{blue}
\nopagenumbers{}

\usepackage[utf8]{inputenc}

% adjust the page margins
\usepackage[scale=0.75]{geometry}

% personal data
\firstname{Antoine}
\familyname{Albertelli}
%\title{Etudiant EPFL}
\address{Ch. de l'Auverney 9}{1814 La Tour-de-Peilz}
\mobile{079~384~38~62}
\phone{021~944~45~44}
\email{antoine.albertelli@epfl.ch}
%\homepage{www.bitbucket.org/antoinealb}

% FIXME I am not so sure about the photo...
% \photo[64pt][0.4pt]{neko}

%----------------------------------------------------------------------------------
%            content
%----------------------------------------------------------------------------------
\begin{document}
%-----       resume       ---------------------------------------------------------
\makecvtitle

\section{Formation}
\cventry{2010--Actuel}{Bachelor}{EPFL}{Lausanne}{}{}
\cventry{2007--2010}{Maturité fédérale}{Gymnase de Burier}{La Tour-de-Peilz}{}{Option application des maths et physique}

\section{Expérience}
\cventry{2011-Actuel}{Assistant}{Bricobot}{Lausanne}{}{Assistant du cours d'introduction à Arduino donné par le Pr. Nicoud, Ex-professeur EPFL}
\cventry{2012}{Logisticien}{Adecco}{Vevey}{}{Gestion d'un entrepôt, réception et envois de marchandises}
\cventry{2011--2012}{Assistant-étudiant}{EPFL}{Lausanne}{}{Assistant au cours d'introduction à la programmation du Pr. Ronan Boulic}

\section{Loisirs}
\cvitem{Robotique}{Participation au concours de robotique Eurobot depuis 2007. Mes responsabilités incluent le développement de notre framework interne en \textsc{c}, comprenant entres autres des librairies d'asservissement, de positionnement ainsi que des drivers bas niveaux. Je m'occupe également de la coordination au sein de l'équipe de développement, notamment des merges et des \textit{code reviews}.}

\cvitem{Jeu de rôle}{Membre de l'association de jeu de rôle Dimension Dé à Montreux depuis 2008.}




\section{Langues}
\cvitemwithcomment{Français}{Langue maternelle}{}
\cvitemwithcomment{Anglais}{Courant}{}
\cvitemwithcomment{Allemand}{Notions}{}

\section{Outils informatiques}
\cvitem{Outils familiers}{\textsc{c}, Mercurial, Linux}
\cvitem{Outils connus}{Python, Solidworks, \LaTeX,\textsc{matlab}, OpenGL}
% For letter example, see revision 1 of repository.

\end{document}

