\documentclass[11pt,a4paper, sans]{moderncv}

\moderncvstyle{casual}  % style options are 'casual' (default), 'classic', 'oldstyle' and 'banking'
\moderncvcolor{grey}
\nopagenumbers{}

\usepackage[utf8]{inputenc}

% adjust the page margins
\usepackage[scale=0.75]{geometry}

% personal data
\firstname{Antoine}
\familyname{Albertelli}
%\title{Etudiant EPFL}
\address{Ch. de l'Auverney 9}{1814 La Tour-de-Peilz}
\mobile{079~384~38~62}
\phone{021~944~45~44}
\email{antoine.albertelli@epfl.ch}
%\homepage{www.bitbucket.org/antoinealb}

% FIXME I am not so sure about the photo...
% \photo[64pt][0.4pt]{neko}

%----------------------------------------------------------------------------------
%            content
%----------------------------------------------------------------------------------
\begin{document}
%-----       resume       ---------------------------------------------------------
\makecvtitle

\section{Formation}
\cventry{2010--Actuel}{Bachelor}{EPFL}{Lausanne}{}{}
\cventry{2007--2010}{Maturité fédérale}{Gymnase de Burier}{La Tour-de-Peilz}{}{}

\section{Expérience}

\cventry{2008--Actuel}{Développeur embarqué C}{CVRA}{Renens}{}{Participation au concours de robotique Eurobot.Mes développements réalisés au sein du club sont: \begin{itemize}
        \item{2008 : Développement d'un automate de tri en utilisant Arduino.}
        \item{2009 : Programmation d'une carte de contrôle pneumatique sous Arduino.}
        \item{2010 : Développement d'une carte éléctronique de régulation autour d'un \textsc{avr} et programmation du code de régulation et pathfinding.}
        \item{2011 : Portage du code existant de \textsc{avr} vers \textsc{niosii} et programmation d'un bras de type \textsc{scara}.}
        \item{2012 (prévisions) : Maintenance du code développé en 2011 et formation des nouveaux membres.}
\end{itemize}
Je m'occupe également de diverses tâches concernant la collaboration entre développeurs, comme l'intégration (\textit{merges}), le contrôle qualité ou l'écriture de documentation.}
\cventry{2011--Actuel}{Assistant}{Bricobot}{Lausanne}{}{Assistant du cours d'introduction à Arduino donné par le Pr. Nicoud, Ex-professeur EPFL}
\cventry{2012}{Logisticien}{Adecco}{Vevey}{}{Gestion d'un entrepôt, réception et envois de marchandises}
\cventry{2011--2012}{Assistant-étudiant}{EPFL}{Lausanne}{}{Assistant au cours d'introduction à la programmation du Pr. Ronan Boulic}

\section{Loisirs}
\cvitem{Jeu de rôle}{Membre de l'association de jeu de rôle Dimension Dé à Montreux depuis 2008.}

\section{Langues}
\cvitemwithcomment{Français}{Langue maternelle}{}
\cvitemwithcomment{Anglais}{Courant}{}
\cvitemwithcomment{Allemand}{Notions}{}

\section{Outils informatiques}
\cvitem{Familiers}{\textsc{c}, Mercurial, Linux}
\cvitem{Connus}{Python, Solidworks, OpenGL, MATLAB, \LaTeX}
% For letter example, see revision 1 of repository.

\end{document}

